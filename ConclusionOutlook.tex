\documentclass[./main.tex]{subfiles}
\begin{document}
\chapter{Conclusion and outlook}
In this thesis, numerous contributions and continued code implementations for the guiding center code \textit{GORILLA} have been presented. The goal was here to provide a detailed insight into the working principles of these developments, such that possible future contributors have additional documentation allowing for an easier understanding of the approach. Since guiding-center codes remain a key component for computing kinetic equilibria as well as quasi-steady plasma parameters in toroidal fusion devices, they underlie stringent requirements, namely, computational efficiency, low sensitivity to noise and physically correct long time orbit dynamics. In order to meet the requirements, the guiding center code \textit{GORILLA} has been initially developed and implemented by M. Eder in cooperation with S.V. Kasilov and C.G. Albert under supervision of W. Kernbichler \cite{Eder_DA}. There, an approach of locally linearizing field quantities on a spatial grid and subsequently integrating the guiding center equations of motion in an iterative Runge Kutta scheme using cylindrical coordinates was chosen. In addition to the \textit{cylindrical contour grid} which had been originally implemented, in this thesis a \textit{field-aligned grid} has been introduced which is suitable for guiding-center orbit integration in symmetry flux coordinates. The use of these coordinates exhibits major advantages with respect to orbit shape and interpolation accuracy of the magnetic vector potential. 
Furthermore, an analytical treatment of the linearized guiding center equations of motion has been presented. On the one hand, this has allowed to derive an analytical expression for the \textit{Runge Kutta 4} error, which has been analyzed and identified as negligible to the accuracy of results obtained by \textit{GORILLA}. On the other hand, this has enabled the implementation of a completely new approach to computing the intersections of the guiding center orbit with the tetrahedral cell boundaries of a given grid element. Here, extensive work has gone into redesign and further development of the previous implementation of the subroutine \texttt{pusher\_tetra\_RK} and also into implementing the new approach based on a polynomial expansion of the analytic solution in subroutine \texttt{pusher\_tetra\_poly}. \\
Finally, an evaluation of the quality of results obtained by the code \textit{GORILLA} and an analysis of its computational efficiency were given. Here, the expected physical results have been reproduced by \textit{GORILLA} to great accuracy showing that the relevant physics is in fact preserved by the method. Furthermore, computational efficiency of \textit{GORILLA} has been shown to be superior by one order of magnitude when compared to a standard RK4/5 integrator with non-linearized fields.
These features are of great importance for the viability of using \textit{GORILLA} in future projects, such as kinetic modeling of plasma equilibria.\\
Lastly, it should be mentioned that a paper (\textit{Geometric integration of guiding-center orbits in piecewise linear toroidal fields}) on the developed integrator has been written and submitted to \textit{Physics of Plasmas}, with the author of this thesis appearing as co-author. At the time of writing up this thesis, this paper is with the editors.

\newpage
\end{document}
\documentclass[./main.tex]{subfiles}
\begin{document}
\chapter{Monte Carlo simulation of particle transport using \texttt{Gorilla}}
possibly shortly describe MC simulation and calculation of diffusion coefficient
%\section{Transport coefficient definition}
%write out what transport coefficients are (D11, etc.), how they are defined, that they are functions of collisionality, introduce $\nu^\ast$
%
%\section{Starting positions for Monte Carlo runs: Normalized flux tube volume}
%explain how for a random starting position the 1/B is integrated up, then at the end it is normalized and a random number from 0 to 1 selects the associated starting position -> mention seed maybe 
%\section{Performing Monte Carlo runs}
%ensemble of particles starting equally distributed on a given flux surface so all $s_i(\tau=0)$ are $s_0$, do run for a set time with collisions, then take a look at the distribution of $s_i(\tau=\tau_{step})$
%\section{Monte Carlo evaluation}
%How to get diffusion and transport coefficients from a distribution of $s_i$.
%\section{Results -  transport coefficients in Tokamak / Hydra Stellarator}
%Take a look at the results 
\newpage
\end{document}
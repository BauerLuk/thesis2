\documentclass[./main.tex]{subfiles}
\addcontentsline{toc}{chapter}{Abstract}
\begin{document}
\chapter*{Kurzfassung}
\label{cha:kurzfassung}
Effiziente Orbitintegration ist ein fundamentaler Baustein für die Berechnung von kinetischen Plasmagleichgewichten und quasi-stationären Plasmaparametern in gyrokinetischen Simulationsprogrammen. Für diese Anwendungen wurde der geometrische Integrator \textit{GORILLA}  (Geometric ORbit Integration with Local Linearisation Approach) im Rahmen dieser Diplomarbeit weiterentwickelt und präsentiert. In \textit{GORILLA} werden die Bewegungen der Gyrationszentren von geladenen Teilchen in lokal linearisierten Feldern in einem tetrahedralen Gitter berechnet. Der beschriebene Integrator besitzt hierbei wünschenswerte Eigenschaften wie hohe Recheneffizienz, geringe Sensibilität auf Ungenauigkeiten von Feldgrößen, sowie eine ausgezeichnete Langzeitstabilität durch die symplektische Formulierung der Bewegungsgleichungen des Gyrationszentrums. Des Weiteren ermöglicht bei \textit{GORILLA} die Vewendung eines Gitters eine sehr effiziente Implementierung von \textit{box counting} Algorithmen, da in \textit{GORILLA} Gyrationsorbits aufgrund der Linearisierung von Feldgrößen stets von einem Zellenrand zum nächsten berechnet werden müssen. In dieser Arbeit werden mehrere Beiträge und Verbesserungen am besprochenen Programm beschrieben. 
Der grundlegendste Beitrag ist hier, dass Berechnungen von Gyrationsorbits auf einem, entlang des Feldes ausgerichteten, Gitter in \textit{symmetry flux coordinates} ermöglicht wurden. Weiters wird die Herleitung der analytischen Lösung der Bewegungsgleichungen in dieser Arbeit präsentiert. Diese ermöglicht einerseits die Beschreibung des Fehlers der Runge Kutta 4 Methode bei der numerischen Integration der Bewegungsgleichungen und eröffnet andererseits neue Zugänge für die Berechnung der Schnittpunkte von Gyrationsorbits mit den Grenzflächen der Gitterelemente entlang des Orbits. Hierbei wurden für die Integration der Gyrationszentren im Rahmen dieser Diplomarbeit mehrere Algorithmen weiterentwickelt, implementiert und präsentiert. 
Um die physikalische Korrektheit der Anwendung zu demonstrieren, wurden weiters Monte Carlo Simulationen durchgeführt um den mono-energetischen radialen Diffusionskoeffizienten zu berechnen. Durch den Vergleich der Ergebnisse von \textit{GORILLA} mit denen der Referenzmethode Runge Kutta 4/5 konnte hierbei gezeigt werden, dass \textit{GORILLA} eine sehr gute Übereinstimmung der Ergebnisse besitzt bei einer um eine Größenordnung höheren Rechengeschwindigkeit.
\end{document}
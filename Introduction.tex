\documentclass[main.tex]{subfiles}
\begin{document}
\chapter{Introduction}
\label{chp:introduction}
\section{Background}
In plasma physics, particularly in kinetic theory, guiding-center codes are an importantant tool in computing quasi-steady plasma parameters by simulating massive amounts of charged particle orbits in toroidal fusion devices. Guiding-centers are hereby the microscopic particle orbits averaged over the fastest time scale, in this case the gyrating motion of the particle. This gyrating motion is induced by the Lorentz force and it shows as an overlying nearly circular motion perpendicular to the magnetic field. Now, by performing massive computations of stochastic guiding-center orbits in an iterative Monte Carlo scheme, the properties of a given system can be evaluated using a box counting approach on a defined grid. The system is hereby described by the single particle distribution function $f_\alpha(\textbf{r},\textbf{v})$ of particle species $\alpha$, which is the probability density for finding a particle at position $\textbf{r}$ with velocity $\textbf{v}$. For a grid element at position $\textbf{r}$, the moments of $f_\alpha(\textbf{r},\textbf{v})$ (i.e., charge density $\rho$ and current density $\textbf{j}$) can be obtained from the averaged guiding-center motion of the particle ensemble through this cell. While the particles of a given ensemble are non-interacting, collisions with the plasma can be introduced artificially by adding random small angle scattering processes (with a Lorentz scattering operator) after each collisionless time step. From the moments of the distribution function, one can moreover self-consistently compute the contributions to the electric and magnetic fields using Maxwell's equations. Doing so, the current field components can be updated according to the new field contributions. By implementing the process of tracing guiding-center positions, sampled from $f_\alpha(\textbf{r},\textbf{v})$, together with the evaluation of the field contributions in an iterative scheme, one aims to model kinetic plasma equilibria. \\
However, due to the stochastic nature of this approach, massive amounts of particles need to be simulated in order to obtain reasonably accurate results. This leads to the requirement of high computational efficiency for a potential integrator.
This stochasticity furthermore generally introduces noise in the evaluated field contributions which can cause unphysical orbit behavior if interpolated fields underlie high order spline oscillations. For this reason, the additional requirement of low sensitivity to noise in field quantities is stated. A third requirement is that the box counting algorithm for guiding center orbits also needs to be efficient in order for kinetic plasma modeling to be feasible.\\
A geometric integrator named \textit{GORILLA} which elegantly satisfies these requirements while preserving physically correct long time orbit dynamics has therefore been developed. 

\section{Overview - GORILLA}
Firstly, this thesis is based on the work of a previous Master's Thesis written by M. Eder \cite{Eder_DA} under supervision of W. Kernbichler and on the contributions made by S.V. Kasilov, C.G. Albert and M. Meisterhofer to this project. At the time of the author joining this project, M. Eder had continued work at the institute on development of the integrator now named \textit{GORILLA}, standing for \textit{Geometric ORbit Integration with Local Linearisation Approach}. For the practical part of this thesis, the author has joined the work group lead by W. Kernbichler in order to further develop the integrator in cooperation with M. Eder.\\
As introduced in \cite{Eder_DA}, the goal of the developed code is to compute guiding-center orbits in toroidal fusion devices with specific requirements on computational efficiency while remaining insensitive to statistical noise. Furthermore the preservation of physically correct long time orbit dynamics is required. For the method of linearizing field components (cf. \cite{Eder_DA}), as well as for the proposed box counting scheme to evaluate particle distribution functions (cf. \cite{Eder_DA}), a grid was introduced which is briefly described as the \textit{cylindrical contour grid} within this thesis. While this previously implemented grid has enabled computation of guiding-center orbits in cylindrical coordinates, Poincar\'e plots of drift surfaces assume polygonal shapes in real space due to the used linearization approach in cylindrical coordinates. Additionally, diffusive behavior had been detected which was initially attributed to the differences in Fourier spectra of magnetic fields within adjacent grid elements. In order to address these issues, the implementation of a new field-aligned grid using symmetry flux coordinates is discussed within this thesis. Here, M. Meisterhofer deserves special mention for his contributions on the implementation of the field aligned grid within the framework of his Bachelor's Thesis at the work group. In order to provide additional documentation of the code, the implementation process of the field aligned grid will be presented on a rather detailed and technical level.\\
Additionally to the implementation of the field-aligned grid, an analytical solution of the linearized set of equations of guiding-center motion is derived within this thesis. This derivation leads to both an analytical description of the Runge Kutta 4 (RK4) error that occurs when numerically solving the equations of guiding-center motion and to the implementation of a polynomial expansion approach for finding intersections of the guiding-center orbit with the tetrahedral cell boundaries along the orbit. Regarding the problem of finding the cell boundary intersections, also the pre-existing subroutine  \texttt{pusher\_tetr\_orb} is reworked and explained within this thesis.\\
Finally, results for Monte Carlo calculations of the mono-energetic radial diffusion coefficient and for benchmarking the computational efficiency of the code are given in this thesis. These results are directly taken from a paper regarding the \textit{GORILLA} project by M. Eder \textit{et al} \cite{paper_gorilla}, for which the author of this thesis has contributed as co-author.



\end{document}
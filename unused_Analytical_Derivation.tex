\documentclass[../main.tex]{subfiles}
\chapter{Derivation of analytical solution to equations of motion}


%\title{Geometrical integrator for guiding center orbits
%in non-axisymmetric fusion devices}

%\author{ 
%%S.~V.~Kasilov$^{1,2}$, A.~M.~Runov$^{3}$\footnote[4]{Corresponding
%%author, runov@ipp.mpg.de, tel. +493834882432}
%}

%\affiliation{
%$^{1}$Institute of Plasma Physics, National Science Center ``Kharkov
%Institute of Physics and Technology'', 61108, Kharkov, Ukraine, \\
%$^{2}$Fusion@\"OAW,\\
%Institut f\"ur Theoretische Physik - Computational Physics,\\
%Technische Universit\"at Graz \\
%Petersgasse~16, A--8010 Graz, Austria
%}

%\affiliation{$^{3}$Max-Planck-Institut f\"ur Plasmaphysik,
%D-17491 Greifswald, Germany}

%\section{INTRODUCTION
%
%\section{Linear equations of motion - Introduction by Sergei}
%\label{sec:linear_equations_sergei}
%
%\noindent
%%Here we use the same idea of linear interpolation for all relevant quantities, but now in 3D.
%Here we split the space into small cells and
%approximate in each cell 
%exact equations of motion by a set of linear ordinary differential equations.
%This linear set has two exact properties: it conserves total energy and magnetic moment and
%ensures that the orbits are divergence free in the coordinate space.
%
%\subsection{Guiding center equations in vector form and in curvilinear variables}
%
%\noindent
%Divergence free (Hamiltonian) form of drift orbits is described by the lines of force of the effective
%Morozov-Solov'ev field,
%\be{morsolB}
%\bB^\ast = \nabla\times\bA^\ast,
%\ee
%where the effective vector potential is expressed through magnetic field parameters and invariants of motion
%$w$ (total energy) and $J_\perp$ (perpendicular adiabatic invariant) as follows
%\be{morsolA}
%\bA^\ast = \bA+\frac{v_\parallel}{\omega_c}\bB,
%\qquad
%v_\parallel=\sigma \left(\frac{2}{m_\alpha}\left(w-J_\perp\omega_c-e_\alpha\Phi\right)\right)^{1/2},
%\ee
%where $\bA$, $\bB$, $\omega_c=e_\alpha B/(m_\alpha c)$ are usual vector-potential, magnetic field strength
%and cyclotron frequency, respectively, $\Phi$ is electrostatic potential, and $\sigma$, $e_\alpha$, 
%$m_\alpha$ and $c$ are parallel velocity sign,
%particle change and mass and the speed of light, respectively. Note that $\bA^\ast=c{\bf P}/e_\alpha$ where
%$\bf P$ is the canonical momentum in guiding center approximation.
%Derivatives in~\eq{morsolB} are computed treating $v_\parallel$ as function of 
%coordinates given by~\eq{morsolA} and treating $w=m_\alpha v^2/2+e_\alpha\Phi$ and 
%$J_\perp=m_\alpha v_\perp^2/(2\omega_c)$ as constants,
%and equations of guiding center motion are
%\be{gcmot}
%\dot \br = v_\parallel \frac{\bB^\ast}{B_\parallel^\ast},
%\qquad
%B_\parallel^\ast = \bh\cdot \bB^\ast,
%\ee
%with $\bh=\bB/B$ being the unit vector along the magnetic field.
%
%\noindent
%In curvilinear variables $x^i$ equations of motion~\eq{gcmot} are
%\be{eqm_curv}
%\dot x^i = \frac{v_\parallel \varepsilon^{ijk}}{\sqrt{g} B_\parallel^\ast}\difp{A^\ast_k}{x^j},
%\ee
%where $g$ is a metric determinant and
%covariant components of effective vector potential $A^ \ast_k$ are expressed via covariant components
%of vector potential $A_k$ and of the magnetic field $B_k$ as follows
%\be{covarcomp}
%A^\ast_k = A_k + \frac{v_\parallel}{\omega_c}B_k.
%\ee
%Introducing the notation
%\be{defU}
%\frac{v_\parallel^2}{2}=U=\frac{1}{m}\left(w-J_\perp\omega_c-e_\alpha\Phi\right)
%\ee
%we can write equation of motion for $v_\parallel$
%\be{vpardot}
%\dot v_\parallel = \frac{1}{v_\parallel}\dot x^i\difp{U}{x^i}
%\ee
%where $U=U(\bx)$ and $\dot x^i$ are given by~\eq{eqm_curv}. Treating $v_\parallel$ as an independent variable
%guiding center equations become a set of four equations
%\bea{setofgceqs}
%B_\parallel^\ast\sqrt{g}\dot x^i = \frac{\rd x^i}{\rd \tau}
%&=&
%\varepsilon^{ijk}\left(
%v_\parallel \difp{A_k}{x^j}+2U\difp{}{x^j}\frac{B_k}{\omega_c}+\frac{B_k}{\omega_c}\difp{U}{x^j}
%\right),
%\nonumber\\
%B_\parallel^\ast\sqrt{g}\dot v_\parallel = \frac{\rd v_\parallel}{\rd \tau}
%&=&
%\varepsilon^{ijk}\difp{U}{x^i}
%\left(
%\difp{A_k}{x^j}+v_\parallel \difp{}{x^j}\frac{B_k}{\omega_c}
%\right).
%\eea
%Here we formally introduced orbit parameter $\tau$ related to time by $\rd t = B_\parallel^\ast\sqrt{g}\rd \tau$.
%If $\tau$ is used as an independent variable of the ODE set, time evolution is obtained in the implicit form
%integrating the above equation for $t(\tau)$ with $B_\parallel^\ast\sqrt{g}$ being a known function of $\tau$.
%In order to reduce this exact equation set to a linear set we notice that
%common factor $B_\parallel^\ast\sqrt{g}$ does not influence the geometry of the orbits but only affects time evolution
%which is not needed very accurately. 
%Therefore this factor can be set to a constant within cell what makes implicit time dependence explicit.
%Approximating quantities 
%$A_k$, $B_k/\omega_c$, $\Phi$ and $\omega_c$ by linear functions~\eq{setofgceqs} becomes a set of four 
%linear equations.
%Note that quantities $A_k$, $B_k/\omega_c$ and $\omega_c$ are linearly interpolated independently from each other.
%Although they are linked with each other, mutual relations contain the metric tensor, so one can assume that
%this tensor is modified in such a way that those relations hold.
%
%\noindent
%Denoting the extended set of variables with $z^i$ where $z^i=x^i$ for $i=1,2,3$ and $z^4=v_\parallel$
%linearized equation set~\eq{setofgceqs} takes a standard form
%\be{standeqset}
%\frac{\rd z^i}{\rd \tau} = a^i_l z^l + b^i,
%\ee
%where for $i,l=1,2,3$ matrix elements are
%\bea{amatdef}
%a^i_l &=& \varepsilon^{ijk}\left(
%2\difp{U}{x^l}\difp{}{x^j}\frac{B_k}{\omega_c}+\difp{U}{x^j}\difp{}{x^l}\frac{B_k}{\omega_c}
%\right) 
%\qquad\mbox{for}\qquad 1\le i,l \le 3,
%\nonumber \\
%a^i_4 &=& \varepsilon^{ijk} \difp{A_k}{x^j}
%\qquad\mbox{for}\qquad 1\le i \le 3,
%\nonumber \\
%a^4_l &=& 0 
%\qquad\mbox{for}\qquad 1\le l \le 3,
%\nonumber \\
%a^4_4 &=& \varepsilon^{ijk}\difp{U}{x^i}
%\difp{}{x^j}\frac{B_k}{\omega_c},
%\eea
%and components of vector $b^i$ are
%\bea{bvecdef}
%b^i &=& \varepsilon^{ijk}\left(
%2U_0\difp{}{x^j}\frac{B_k}{\omega_c}+\left(\frac{B_k}{\omega_c}\right)_0\difp{U}{x^j}
%\right)
%\qquad\mbox{for}\qquad 1\le i \le 3,
%\nonumber \\
%b^4 &=& \varepsilon^{ijk}\difp{U}{x^i}
%\difp{A_k}{x^j}.
%\eea
%Here, quantities with zero mean the value at the origin of the coordinates,
%\be{valorig}
%U=U_0+x^i\difp{U}{x^i}, \qquad \frac{B_k}{\omega_c} = \left(\frac{B_k}{\omega_c}\right)_0
%+x^i \difp{}{x^i}\frac{B_k}{\omega_c}.
%\ee
%
%\subsection{Straightforward solution of the linear set}
%
%\noindent
%As a first step, we transform the inhomogeneous equation set~\eq{standeqset} to a homogeneous set
%by shifting the unknowns,
%\be{shiftz}
%z^i=\tilde z^i+z_s^i,
%\ee
%where $\tilde z^i$ are new unknowns and $z_s^i$ is constant shift which satisfies
%\be{eqforshift}
%a^i_l z_s^l + b^i=0.
%\ee
%Equation set for shifted variables is
%\be{standeqset_hom}
%\frac{\rd \tilde z^i}{\rd \tau} = a^i_l \tilde z^l.
%\ee
%Introducing the eigenvectors $\varphi^i_{k}$ of matrix $a^i_l$,
%\be{eigvecval}
%a^i_l \varphi^l_{\bar k}=\lambda_{\bar k} \varphi^i_{\bar k}
%\ee
%where $\bar k$ means that there is no summation convention over this index, and $\lambda_{\bar k}$ are corresponding 
%eigenvalues we obtain the solution as
%\be{obtsol}
%\tilde z^i(\tau)=\sum_{\bar k=1}^4 \zeta^{\bar k} \varphi^i_{\bar k} {\rm e}^{\lambda_{\bar k}\tau}.
%\ee
%Here coefficients $\zeta^{\bar k}$ correspond to the initial value of $\tilde z^i$ and are obtained from the solution
%of
%\be{initval}
%\sum_{\bar k=1}^4 \varphi^i_{\bar k} \zeta^{\bar k} = \tilde z^i(0).
%\ee
%Introducing the inverse matrix $\bar \varphi^{\bar k}_i$ to the matrix of eigenvectors $\varphi^i_{\bar k}$
%such that
%\be{invmat}
%\bar \varphi^{\bar k}_i \varphi^i_{\bar k^\prime}=\delta^{\bar k}_{\bar k^\prime},
%\ee
%coefficients $\zeta^{\bar k}$ are explicitly given by
%\be{zetabar}
%\zeta^{\bar k} = \bar \varphi^{\bar k}_i \tilde z^i(0).
%\ee
%
%\subsection{Pre-calculation and storage}
%
%\noindent
%At this point we are interested in massive computation of collisionless orbits where computation speed
%is critical. Therefore various matrices and vectors introduced in the above subsection must be computed
%before orbit computation and stored for each cell. Computation of orbits would consist in mapping of
%the orbit footprint from the entry face of the tetrahedron cell to some exit face which then will be the entry
%face of the next tetrahedron cell. Namely, we would look for values of orbit parameter $\tau$ which correspond
%to crossing of the plane containing a particular face and choose the smallest of these times. 
%Equation of a such plane $m$ where $1\ge m \ge 4$ has the form
%\be{planeeq}
%x^i \alpha_i^{(m)}+\beta^{(m)}=0.
%\ee
%Respectively, equation to be solved for finding the intersection is of the form
%\be{inters}
%\sum_{\bar k=1}^4 \Omega_{\bar k}^{(m)} {\rm e}^{\lambda_{\bar k}\tau}+\rho^{(m)}=0,
%\ee
%where
%\bea{Om_rho}
%\Omega_{\bar k}^{(m)} &=& \alpha_i^{(m)} \varphi^i_{\bar k}\zeta^{\bar k}=\omega_{\bar k}^{(m)} \zeta^{\bar k},
%\qquad \omega_{\bar k}^{(m)} = \alpha_i^{(m)} \varphi^i_{\bar k},
%\nonumber \\
%\rho^{(m)} &=& \beta^{(m)}+x_s^i \alpha_i^{(m)}.
%\eea
%Thus, besides quantities from previous subsection we need to store also $\omega_{\bar k}^{(m)}$ and $\rho^{(m)}$.
%
%\subsection{More detailed solution of the linear set}
%
%\noindent
%For checking the general solution and later on for the approximate solution which can be used as zero order
%approximation for the numerical solution of Eq.~\eq{inters}
%we can solve the linear equation set in a more handy way.
%This way also shows that two of the eigenvalues differ from each other by factor 2 what means one 
%exponent less to compute.
%We notice that equation for $v_\parallel$ does not depend on coordinates because within linear approximation
%$$
%\varepsilon^{ijk}\difp{U}{x^i}\difp{}{x^j}\frac{B_k}{\omega_c}=a,
%\qquad
%\varepsilon^{ijk}\difp{U}{x^i}\difp{A_k}{x^j}=b,
%$$
%where $a$ and $b$ are constants. This equation is then
%$$
%\frac{\rd v_\parallel}{\rd \tau}=a v_\parallel + b
%$$
%with a solution
%$$
%v_\parallel(\tau)=\left(v_\parallel(0)+\frac{b}{a}\right){\rm e}^{a\tau}-\frac{b}{a}.
%$$
%Expressing $U$ via $v_\parallel$ in the first three equations~\eq{setofgceqs} they can be presented as
%\be{eqsforxi}
%\frac{\rd x^i}{\rd \tau}=\tilde{a}^i_l x^l+q^i
%\ee
%where
%\be{gammadef}
%\tilde{a}^i_l=\varepsilon^{ijk}\difp{U}{x^j}\difp{}{x^l}\frac{B_k}{\omega_c}
%\ee
%is a constant matrix
%and the driving term $q^i$ is a known function of $\tau$
%\be{defqi}
%q^i=q^i(\tau)=
%\varepsilon^{ijk}\left(
%v_\parallel(\tau) \difp{A_k}{x^j}+v_\parallel^2(\tau)\difp{}{x^j}\frac{B_k}{\omega_c}
%+\left(\frac{B_k}{\omega_c}\right)_0\difp{U}{x^j}
%\right).
%\ee
%Subscript ``0'' in the last term denotes the value at the coordinate origin 
%such that 
%$$
%\frac{B_k}{\omega_c} = \left(\frac{B_k}{\omega_c}\right)_0+x^l\difp{}{x^l}\frac{B_k}{\omega_c}.
%$$
%By variable change 3D ODE set~\eq{eqsforxi} can be reduced to a 2D set if one of $x^i$ is replaced
%by
%\be{defy}
%y=x^i\difp{U}{x^i}.
%\ee
%From definition~\eq{gammadef} one can see that
%$$
%\difp{U}{x^i} \tilde{a}^i_l=0.
%$$
%Therefore, multiplying~\eq{eqsforxi} with a constant vector $\partial U /\partial x^i$ one gets
%\be{odefory}
%\frac{\rd y}{\rd \tau}=\difp{U}{x^i}q^i(\tau)
%\ee
%which can be directly integrated. This integration however is not needed because
%using the definition~\eq{defU} and linear representation of $U$,
%$$
%U=U_0+x^i\difp{U}{x^i}=U_0+y,
%$$
%one can express
%$$
%y=\frac{1}{2}v_\parallel^2(\tau)-U_0.
%$$
%
%\clearpage
%\section{Symplectic integrator}
%\subsection{Hamiltonian mechanics in non-canonical coordinates}
%
%This section gives a short overview of Hamiltonian mechanics in non-canonical
%coordinates $\zset$. These equations of motion can be equivalently
%obtained by Euler-Lagrange equations of the phase-space Lagrangian
%written in terms of $\zset$ (as in paper of Littlejohn), or by using
%the Jacobian matrix of transformation from the canonical Hamiltonian
%formalism (as in book of Balescu). Here we take the latter path. Take
%equations of motion in canonical coordinates $\zset_{\mathrm{c}}=(\qset,\pset)=(q^{1},q^{2},\dots q^{N},p_{1},p_{2},\dots p_{N})=(z_{c}^{\,\alpha})_{\alpha=1\dots2N}$,
%\begin{align}
%\dot{q}^{i} & =\frac{\partial H}{\partial p_{i}},\\
%\dot{p}_{j} & =-\frac{\partial H}{\partial q^{j}}.
%\end{align}
%The notation with $i$ and $j$ up and down, but greek indices $\alpha,\beta$
%always up comes from the fact that momentum coordinates describe components
%of covectors (index down) in velocity space, but also general phase-space
%coordinates (index up). More compactly equations of motion can be
%written as
%\begin{equation}
%\dot{z}_{\mathrm{c}}^{\,\alpha}=J^{\alpha\beta}\frac{\partial H}{\partial z_{\mathrm{c}}^{\,\beta}},
%\end{equation}
%where 
%\begin{equation}
%(J^{\alpha\beta})=\left(\begin{array}{cc}
%0 & (\delta_{i}^{j})\\
%(-\delta_{j}^{i}) & 0
%\end{array}\right)
%\end{equation}
%is an antisymmetric matrix with positive and negative identity matrix
%in the upper right and lower left block, respectively. Equivalently,
%by inverting $\v J$, or just switching order and a minus sign in
%canonical equations of motion,
%\begin{align}
%-\dot{p}_{j} & =\frac{\partial H}{\partial q^{j}},\\
%\dot{q}^{i} & =\frac{\partial H}{\partial p_{i}},
%\end{align}
%we can write
%\begin{equation}
%\bar{J}_{\alpha\beta}\dot{z}_{\mathrm{c}}^{\,\alpha}=\frac{\partial H}{\partial z_{\mathrm{c}}^{\,\beta}},
%\end{equation}
%where
%\begin{equation}
%(\bar{J}_{\alpha\beta})=\left(\begin{array}{cc}
%0 & (-\delta_{i}^{j})\\
%(\delta_{j}^{i}) & 0
%\end{array}\right).
%\end{equation}
%Now if we switch to non-canonical coordinates $\zset$ the chain rule
%tells us that
%\begin{align}
%\dot{z}_{\mathrm{c}}^{\,\gamma} & =\frac{\partial z_{\mathrm{c}}^{\,\gamma}}{\partial z^{\alpha}}\dot{z}^{\alpha},\\
%\frac{\partial H}{\partial z_{\mathrm{c}}^{\delta}} & =\frac{\partial z^{\beta}}{\partial z_{\mathrm{c}}^{\,\delta}}\frac{\partial H}{\partial z^{\beta}}.
%\end{align}
%Thus we can write equations of motion in terms of non-canonical coordinates
%as
%\begin{align}
%\frac{\partial z_{\mathrm{c}}^{\,\gamma}}{\partial z^{\alpha}}\dot{z}^{\alpha} & =J^{\gamma\delta}\frac{\partial z^{\beta}}{\partial z_{\mathrm{c}}^{\,\delta}}\frac{\partial H}{\partial z^{\beta}}.
%\end{align}
%This can be treated in two ways. If we multiply by a left inverse
%of the Jacobian matrix of the coordinate transform we obtain
%\begin{align}
%\dot{z}^{\alpha} & =\frac{\partial z^{\alpha}}{\partial z_{\mathrm{c}}^{\,\gamma}}J^{\gamma\delta}\frac{\partial z^{\beta}}{\partial z_{\mathrm{c}}^{\,\delta}}\frac{\partial H}{\partial z^{\beta}}.
%\end{align}
%This is of an uncoupled Hamiltonian form
%\begin{equation}
%\dot{z}^{\alpha}=\Lambda^{\alpha\beta}\frac{\partial H}{\partial z^{\beta}}
%\end{equation}
%with the antisymmetric \emph{Poisson matrix}
%\begin{equation}
%\Lambda^{\alpha\beta}\equiv\frac{\partial z^{\alpha}}{\partial z_{\mathrm{c}}^{\,\gamma}}J^{\gamma\delta}\frac{\partial z^{\beta}}{\partial z_{\mathrm{c}}^{\,\delta}}.
%\end{equation}
%By taking the inverse we obtain (covariant) components of the antisymmetric
%\emph{symplectic form}
%\begin{equation}
%(\omega_{\alpha\beta})=(\Lambda^{\alpha\beta})^{-1}
%\end{equation}
%and coupled equations of motion
%\begin{equation}
%\omega_{\alpha\beta}\dot{z}^{\beta}=\frac{\partial H}{\partial z^{\alpha}}.
%\end{equation}
%In terms of tensor calculus in phase-space, the Poisson matrix $\Lambda^{\alpha\beta}$
%represents the Poisson tensor in contravariant form, and $\omega_{\alpha\beta}$
%represents the symplectic 2-form. Both are antisymmetric and follow
%usual transformation properties. In the special case of canonical
%coordinates, $\Lambda^{\alpha\beta}=J^{\alpha\beta}$ and $\omega_{\alpha\beta}=\bar{J}_{\alpha\beta}$
%are block diagonal antisymmetric matrices. In the general non-canonical
%case, $\Lambda^{\alpha\beta}$ and $\omega_{\alpha\beta}$ depend
%on phase-space coordinates $\zset$, but not on time $t$.
%
%Geometric integrators that solve equations of motion exactly are symplectic
%per definition (2D: Kasilov/Runov and 3D: Eder/Kasilov). For the construction
%of numerical symplectic integrators with non-canonical quadrature,
%certain tricks are required (Albert/Kasilov). Another alternative
%is to keep the formulation with a (degenerate) phase-space Lagrangian
%and use stabilized variational integrators (Kraus).
%
%\subsection{Geometric 3D integrator}
%
%A detailed description of the integrator can be found in the master's
%thesis of Michael Eder {[}1{]} who is the main developer of the mesh-based
%geometric orbit integration code.
%
%Equations of {[}1{]} can be re-written using an antisymmetric Poisson
%matrix. Namely we have equations of motion (2.9) with
%\begin{align}
%\frac{\d x^{i}}{\d\tau} & =\varepsilon^{ijk}\left(z^{4}\frac{\partial A_{k}}{\partial x^{j}}+2U\frac{\partial}{\partial x^{j}}\frac{B_{k}}{\omega_{c}}+\frac{B_{k}}{\omega_{c}}\frac{\partial U}{\partial x^{j}}\right),\\
%\frac{\d v_{\parallel}}{\d\tau} & =\varepsilon^{ijk}\frac{\partial U}{\partial x^{i}}\left(\frac{\partial A_{k}}{\partial x^{j}}+v_{\parallel}\frac{\partial}{\partial x^{j}}\frac{B_{k}}{\omega_{c}}\right).
%\end{align}
%The idea is to use linear interpolation of $A_{k},B_{k}/\omega_{c}$
%and $U$ independently to obtain a linear set of equations of motion
%in each mesh element. Covariant components $h_{k}=B_{k}/B=e/mc\,B_{k}/\omega_{c}$
%of the normalized guiding field can be used to compute of $B_{k}/\omega_{c}$,
%and $U=(w-J_{\perp}\omega_{c}-e\Phi)$ is computed from linear interpolations
%of $\omega_{c}$ and $\Phi$.
%
%\subsubsection*{Trying to bring the system to Poisson form}
%
%Now the question is what to move into the Poisson matrix, and what
%into the Hamiltonian. Since we use $U$ and $v_{\parallel}$ we have
%some freedom to move things around. Namely in the exact system
%\begin{align}
%H & =\frac{mv_{\parallel}^{\,2}}{2}+J_{\perp}\omega_{c}(\xset)+e\Phi(\xset)\\
% & =\frac{mv_{\parallel}^{\,2}}{2}+w-mU(\xset)\\
% & =w.
%\end{align}
%Using the middle form we obtain the following gradient of $H$ in
%phase-space,
%\begin{align}
%\frac{\partial H}{\partial x^{i}} & =-m\frac{\partial U}{\partial x^{i}},\\
%\frac{\partial H}{\partial v_{\parallel}} & =mv_{\parallel}.
%\end{align}
%Now we want to get equations of motion
%\begin{equation}
%\dot{z}^{\alpha}=\Lambda^{\alpha\beta}\frac{\partial H}{\partial z^{\beta}}
%\end{equation}
%with an antisymmetric $\Lambda^{\alpha\beta}$, where $\zset=(x^{1},x^{2},x^{3},v_{\parallel})$,
%and we use Greek indices to go from $1\dots4$ rather than $1\dots3$
%for Latin ones. Antisymmetry of $\Lambda^{\alpha\beta}$ guarantees
%symplecticity, i.e. conservation of the symplectic form, and thereby
%phase-space volume (Liouville's theorem) and conservation of invariants.
%We want to obtain equations of motion (2.8) as
%\begin{align}
%\dot{z}^{i}=\dot{x}^{i} & =\frac{1}{B_{\parallel}^{\star}\sqrt{g}}\varepsilon^{ijk}\left(v_{\parallel}\frac{\partial A_{k}}{\partial x^{j}}+2U\frac{\partial}{\partial x^{j}}\frac{B_{k}}{\omega_{c}}+\frac{B_{k}}{\omega_{c}}\frac{\partial U}{\partial x^{j}}\right),\label{eq:xdot}\\
%\dot{z}^{4}=\dot{v}_{\parallel} & =\frac{1}{B_{\parallel}^{\star}\sqrt{g}}\varepsilon^{ijk}\frac{\partial U}{\partial x^{i}}\left(\frac{\partial A_{k}}{\partial x^{j}}+v_{\parallel}\frac{\partial}{\partial x^{j}}\frac{B_{k}}{\omega_{c}}\right).\label{eq:vpardot}
%\end{align}
%For sure we want
%\begin{align}
%\Lambda^{4i} & =-\frac{1}{mB_{\parallel}^{\star}\sqrt{g}}\varepsilon^{ijk}\left(\frac{\partial A_{k}}{\partial x^{j}}+v_{\parallel}\frac{\partial}{\partial x^{j}}\frac{B_{k}}{\omega_{c}}\right),\\
%\Lambda^{44} & =0.
%\end{align}
%This way the equation (\ref{eq:vpardot}) in $\dot{v}_{\parallel}$
%is fulfilled. Antisymmetry requires that
%\begin{equation}
%\Lambda^{i4}=-\Lambda^{4i}=+\frac{1}{mB_{\parallel}^{\star}\sqrt{g}}\varepsilon^{ijk}\left(\frac{\partial A_{k}}{\partial x^{j}}+v_{\parallel}\frac{\partial}{\partial x^{j}}\frac{B_{k}}{\omega_{c}}\right).
%\end{equation}
%If we choose in addition
%\begin{equation}
%\Lambda^{ij}=-\frac{\varepsilon^{ijk}}{mB_{\parallel}^{\star}\sqrt{g}}\frac{B_{k}}{\omega_{c}}
%\end{equation}
%we obtain 
%\begin{equation}
%\dot{x}=\Lambda^{i4}\frac{\partial H}{\partial v_{\parallel}}+\Lambda^{ij}\frac{\partial H}{\partial x^{j}}=\frac{1}{B_{\parallel}^{\star}\sqrt{g}}\varepsilon^{ijk}\left(v_{\parallel}\frac{\partial A_{k}}{\partial x^{j}}+v_{\parallel}^{\ 2}\frac{\partial}{\partial x^{j}}\frac{B_{k}}{\omega_{c}}+\frac{B_{k}}{\omega_{c}}\frac{\partial U}{\partial x^{j}}\right).
%\end{equation}
%Now this is the correct expression (\ref{eq:xdot}) if $v_{\parallel}^{\,2}=2U$
%exactly during the whole procedure. This is true due to definition
%(2.6) and the time evolution (2.7) of $v_{\parallel}$ based upon
%this equation. Any approximation that does not fulfil (2.6) up to
%computer accuracy will violate symplecticity.
%
%
%
%\clearpage
\section{Analytical solution of linear equations of motion}
\noindent
In this section, an analytic solution to the linear equations of motion derived in section \ref{sec:linear_equations_sergei} will be deduced.\\
By denoting the extended set of variables with $z^i$ where $z^i=x^i$ for $i=1,2,3$ and $z^4=v_\parallel$ the linearized equation set takes a standard form
\be{standeqset}
\frac{\rd z^i(\tau)}{\rd \tau} = a^i_l(\tau) z^l(\tau) + b^i,
\ee
where for $i,l=1,2,3$ the matrix elements are
\bea{amatdef}
a^i_l &=& \varepsilon^{ijk}\left(
2\difp{U}{x^l}\difp{}{x^j}\frac{B_k}{\omega_c}+\difp{U}{x^j}\difp{}{x^l}\frac{B_k}{\omega_c}
\right) 
\qquad\mbox{for}\qquad 1\le i,l \le 3,
\nonumber \\
a^i_4 &=& \varepsilon^{ijk} \difp{A_k}{x^j}
\qquad\mbox{for}\qquad 1\le i \le 3,
\nonumber \\
a^4_l &=& 0 
\qquad\mbox{for}\qquad 1\le l \le 3,
\nonumber \\
a^4_4 &=& \varepsilon^{ijk}\difp{U}{x^i}
\difp{}{x^j}\frac{B_k}{\omega_c},
\eea
and the components of vector $b^i$ are
\bea{bvecdef}
b^i &=& \varepsilon^{ijk}\left(
2U_0\difp{}{x^j}\frac{B_k}{\omega_c}+\left(\frac{B_k}{\omega_c}\right)_0\difp{U}{x^j}
\right)
\qquad\mbox{for}\qquad 1\le i \le 3,
\nonumber \\
b^4 &=& \varepsilon^{ijk}\difp{U}{x^i}
\difp{A_k}{x^j}.
\eea

\subsection{Reduction to a set of three linear ODEs}
\noindent

One can start by looking at the fourth component which is the parallel velocity as a function of time. Since this equation is decoupled from $x^i$ it can be solved independently, resulting to

\noindent
\be{defvpar}
v_\parallel(\tau)=\left(v_\parallel(0)+\frac{b}{a}\right){\rm e}^{a\tau}-\frac{b}{a},
\ee
with $a$ and $b$ being
$$
\varepsilon^{ijk}\difp{U}{x^i}\difp{}{x^j}\frac{B_k}{\omega_c}=a,
\qquad
\varepsilon^{ijk}\difp{U}{x^i}\difp{A_k}{x^j}=b,
$$

which are constant within the linear fields.

For further calculation this expression will be abbreviated using $\eta = (v_\parallel(0)+\frac{b}{a})$ and $\theta = (\frac{b}{a})$:
\be{defvpar2}
v_\parallel(\tau)=\eta{\rm e}^{a\tau}-\theta.
\ee


The set of differential equations can now be formulated in a way that all time dependence is confined to a driving term. In order to do this one starts with equation \ref{standeqset} and takes only the first three components into account, which yields

\bea{derivation of 3d set of equations}
 i &=& 1,2,3 \\
\frac{\textrm{d} x^i(\tau)}{\textrm{d}\tau} &=& a^i_lx^l(\tau) + a^i_4z^4(\tau) + b^i \\
 &=& \underbrace{\varepsilon^{ijk}\left(2 \frac{\partial U}{\partial x^l}\frac{\partial}{\partial x^j}\frac{B_k}{\omega_c} + \frac{\partial U}{\partial x^j}\frac{\partial}{\partial x^l}\frac{B_k}{\omega_c}	\right)}_{=a^i_l}x^l(\tau) + \underbrace{\varepsilon^{ijk} \left( \frac{\partial A_k}{\partial x^j} \right)}_{= a^i_4} \underbrace{z^4(\tau)}_{=v_\parallel(\tau)}  
 \nonumber \\
 &+& \underbrace{\varepsilon^{ijk} \left( 2U_0 \frac{\partial}{\partial x^j}\frac{B_k}{\omega_c}+\left( \frac{B_k}{\omega_c}\right)_0\frac{\partial U}{\partial x^j} \right)}_{=b^i} .
\eea

Using $U(x^i)= U_0 + x^i\frac{\partial U}{\partial x^i}$  and $U = \frac{v_\parallel ^2}{2}$ simplifies this equation to
\be{3d explicit DE for x}
\frac{\textrm{d} x^i(\tau)}{\textrm{d}\tau} = \underbrace{\varepsilon^{ijk}\difp{U}{x^j}\difp{}{x^l}\frac{B_k}{\omega_c}}_{=\tilde{a}^i_l}x^l(\tau)+ \underbrace{v_\parallel(\tau) \varepsilon^{ijk}\frac{\partial A_k}{\partial x^j} + v_\parallel^2(\tau)\varepsilon^{ijk}\frac{\partial}{\partial x^j}\frac{B_k}{\omega_c} + \varepsilon^{ijk} \left( \frac{B_k}{\omega_c}\right)_0 \frac{\partial U}{\partial x^j}}_{q^i(\tau)}.
\ee

This can be compactly written as

\be{short 3d de for x}
\frac{\textrm{d} x^i(\tau)}{\textrm{d}\tau} = \tilde{a}^i_l x^l(\tau) + q^i(\tau),
\ee
 
where
\be{gammadef}
\tilde{a}^i_l=\varepsilon^{ijk}\difp{U}{x^j}\difp{}{x^l}\frac{B_k}{\omega_c}
\ee
is a constant matrix
and the driving term $q^i(\tau)$ is explicitly given as 
\be{defqi}
q^i(\tau)=
\varepsilon^{ijk}\left(
v_\parallel(\tau)  \difp{A_k}{x^j}  +v_\parallel^2(\tau)\difp{}{x^j}\frac{B_k}{\omega_c}
+\left(\frac{B_k}{\omega_c}\right)_0\difp{U}{x^j}
\right).
\ee
For compactness, this will be abbreviated as
\bea{defqi2}
q^i(\tau)&=&v_\parallel(\tau) \underbrace{\varepsilon^{ijk} \left( \difp{A_k}{x^j}\right)}_{\alpha^i}  
+v_\parallel^2(\tau)   \underbrace{ \varepsilon^{ijk}       \difp{}{x^j}\frac{B_k}{\omega_c}}_{\beta^i} 
+ \underbrace{\varepsilon^{ijk} \left( \frac{B_k}{\omega_c}\right)_0\difp{U}{x^j}}_{\gamma^i} \\
&=&v_\parallel(\tau){\alpha^i}+v_\parallel^2(\tau) {\beta^i}+{\gamma^i}.
\eea
The terms $\alpha^i$, $\beta^i$ and $\gamma^i$ are constant within the linearized field, all time dependence comes therefore from $v_\parallel(\tau)$.


\subsection{Solution of Homogeneous Equation of Motion}
\noindent
The next step of calculating the analytical solution to the homogeneous part of equation \ref{short 3d de for x} is to start with the following Ansatz:
\be{Ansatzeq}
\vec{x} = e^{\lambda\tau}\vec{\psi}
\ee
Using this, equation \ref{standeqset} yields the eigenvalue equation
\be{Ansatzeq2}
\lambda e^{\lambda \tau}\vec{\psi} = \hat{a} e^{\lambda\tau}\vec{\psi}.
\ee
In this context $\lambda$ denotes the eigenvalues and $\vec{\psi}$ the eigenvectors of $\hat{a}$.
Therefore, the general solution to the homogeneous differential equation can be written as:
\be{eq:homogeneousSolution1}
\vec{x}_{(h)}(\tau) = C_1e^{\lambda_1\tau}\vec{\psi_1} + C_2e^{\lambda_2\tau}\vec{\psi_2} + C_3e^{\lambda_3\tau}\vec{\psi_3}
\ee
Here, the $C_l$ denote a vector of arbitrary constants given by initial conditions of the problem.\\
Equation \ref{eq:homogeneousSolution1} can be written by using index notation
\be{}
x^i_{(h)}(\tau) = \psi^i_l C^l e^{\lambda^l \tau},
\ee
where $\psi^i_l$ is a matrix with the eigenvectors $\vec{\psi}_l$ in the columns corresponding to the respective eigenvalues $\lambda^l$.

\subsection{Particular Solution: Variation of Constants}
\noindent
Next, the particular solution to the inhomogeneous differential equation \ref{standeqset} will be derived. In order to do this, the method of Variation of Constants is applied. With this approach, the coefficients $\widetilde{C}_l$ are treated as functions of $\tau$
\be{homogeneousSolution2}
\vec{x}_{(p)}(\tau) = \widetilde{C}_1(\tau)e^{\lambda_1\tau}\vec{c_1} + \widetilde{C}_2(\tau)e^{\lambda_2\tau}\vec{c_2} + \widetilde{C}_3(\tau)e^{\lambda_3\tau}\vec{c_3},
\ee
and can also be denoted in index notation
\be{}
x^i_{(p)}(\tau) = \psi^i_l \widetilde{C}^l(\tau) e^{\lambda^l \tau}.
\ee
Calculating the derivative of this equation yields
\be{Variation of Constants}
\frac{\mathrm{d}\vec{x}} {\mathrm{d}\tau} = {\widetilde{C}_1}^\prime(\tau)e^{\lambda_1\tau}\vec{\psi_1}+ \widetilde{C}_1(\tau)\lambda_1e^{\lambda_1\tau}\vec{\psi_1} + {\widetilde{C}_2}^\prime(\tau)e^{\lambda_2\tau}\vec{\psi_2} +\widetilde{C}_2(\tau)\lambda_2e^{\lambda_2\tau}\vec{\psi_2} + {\widetilde{C}_3}^\prime(\tau)e^{\lambda_3\tau}\vec{\psi_3}+ \widetilde{C}_3(\tau)\lambda_3e^{\lambda_3\tau}\vec{\psi_3}
\ee
Inserting this expression into equation \ref{eqsforxi} leads to
\be{VoC2}
 \vec{q}(\tau)  = {\widetilde{C}_1}^\prime(\tau)e^{\lambda_1\tau}\vec{\psi_1} + {\widetilde{C}_2}^\prime(\tau)e^{\lambda_2\tau}\vec{\psi_2}  + {\widetilde{C}_3}^\prime(\tau)e^{\lambda_3\tau}\vec{\psi_3},
\ee
which can be denoted again in index notation
\be{}
q^i(\tau) = \psi^i_l {\widetilde{C}}^{'l}(\tau) e^{\lambda^l \tau}.
\ee
The next step in calculating the coefficients ${\widetilde{C}}^{l}(\tau)$ is to multiply the eigenvectors $\psi^i_l$ with the inverse matrix $\bar{\psi}^j_i$, where $\bar{\psi}^j_i \psi^i_l = \delta^j_l$
\be{}
\bar{\psi}^j_i q^i(\tau) =  \underbrace{\bar{\psi}^j_i \psi^i_l}_{ \delta^j_l} {\widetilde{C}}^{'l}(\tau) e^{\lambda^l \tau}.
\ee
\be{C_i-prime}
\widetilde{C}^{'l}(\tau) =   \bar{\psi}^l_i q^i(\tau) e^{-\lambda^l \tau}
\ee
One can now integrate this expression from $0$ to $\tau$ since the initial conditions will be given for $\tau = 0$:
\be{C_i}
\widetilde{C}^{l}(\tau) = \int_0^\tau \bar{\psi}^l_i q^i(\tau^\prime) e^{-\lambda^l \tau^\prime} \mathrm{d}\tau^\prime\\
\ee
By inserting this into equation \ref{homogeneousSolution2}, the particular solution to the inhomogeneous differential equation \ref{standeqset} is obtained:
\be{x_part_pre_integral}
x^i_{(p)}(\tau) = \psi^i_l e^{\lambda^l \tau} \int_0^\tau \bar{\psi}^l_k q^k(\tau^\prime) e^{-\lambda^l \tau^\prime} \mathrm{d}\tau^\prime
\ee
The general solution to this equation is the superposition of the homogeneous solution with a particular solution, which in this case can be written as
\be{x_total}
x_{g}^i(\tau) =\psi^i_l e^{\lambda^l \tau} \left(C^l + \int_0^\tau \bar{\psi}^l_k q^k(\tau^\prime) e^{-\lambda^l \tau^\prime} \mathrm{d}\tau^\prime \right)
\ee
Next, one can derive an integratable expression for $q^k$. This can be achieved by inserting equation \ref{defvpar2} into equation \ref{defqi2}
\be{qi_short1}
q^k(\tau) = \left( e^{a\tau}\eta -\theta\right)\alpha^k + \left( e^{a\tau}\eta -\theta \right)^2 \beta^k+\gamma^k.
\ee
This expression can be re-written collecting powers of $e^{a\tau}$, it is also convenient to abbreviate this further as
\be{qi_short2}
q^k(\tau) =  e^{a\tau}(\underbrace{\eta \alpha^k - 2 \eta \theta \beta^k}_{D^k}) + e^{2a\tau} (\underbrace{\eta^2 \beta^k}_{F^k}) + (\underbrace{\theta^2 \beta^k-\theta \alpha^k + \gamma^k}_{E^k}),
\ee
which leads to
\be{qi_veryshort}
q^k(\tau) = e^{a\tau}D^k + e^{2a\tau}F^k + E^k.
\ee
Now, one can put this expression into equation \ref{x_total} and thereby obtains
\be{qi_into_xtot}
x^i(\tau) = \psi^i_l e^{\lambda^l \tau} \left( C^l  + \int_0^\tau \left( e^{(a-\lambda^l)\tau^\prime}\bar{\psi}^l_k D^k + e^{(2a-\lambda^l)\tau^\prime} \bar{\psi}^l_k F^k + e^{-\lambda^l\tau^\prime}\bar{\psi}^l_k E^k   \right)d\tau^\prime \right).
\ee
The individual integrals over $\tau^\prime$ yield
\bea{integrals_dtau-prime}
\int_{0}^{\tau} e^{(a-\lambda^l)\tau^\prime}d\tau^\prime = \frac{1}{a-\lambda^l} (e^{(a-\lambda^l)\tau}-1)\\
\int_{0}^{\tau} e^{(2a-\lambda^l)\tau^\prime}d\tau^\prime = \frac{1}{2a-\lambda^l} (e^{(2a-\lambda^l)\tau}-1)\\
\int_{0}^{\tau} e^{-\lambda^l\tau^\prime}d\tau^\prime = -\frac{1}{\lambda^l} (e^{-\lambda^l\tau}-1)
\eea
Using these results, equation \ref{qi_into_xtot} can be re-written as:
\be{x_total2}
x^i(\tau) =   \psi^i_l  \left(C^l e^{\lambda^l \tau} + \frac{\bar{\psi}^l_k D^k}{a-\lambda^l}(e^{a\tau}-e^{\lambda^l\tau}) + \frac{\bar{\psi}^l_k F^k}{2a-\lambda^l}(e^{2a\tau}-e^{\lambda^l\tau})  - \frac{\bar{\psi}^l_k E^k}{\lambda^l}(1-e^{\lambda^l\tau}) \right)
\ee 
Next, the value of $C^l$ need to be determined for given initial conditions
\be{initial_conditions}
x^i(\tau=0) = x^i_{(0)}.
\ee
Setting $\tau = 0$ in equation \ref{x_total2} yields
\be{x0=cjx0}
x^i_{(0)} =  \psi^i_l  C^l + 0 + 0 + -0  = \psi^i_l  C^l .
\ee 
By multiplying with the inverse matrix $\bar{\psi}^l_i$, the coefficients are given by
\be{C_k1}
C^l = \bar{\psi}^l_i x^i_{(0)} .
\ee
Therefore, formula \ref{x_total2} can be re-written as
\be{x_total3}
x^i(\tau) =   \psi^i_l  \left(\bar{\psi}^l_i x^i_{(0)}e^{\lambda^l \tau} + \frac{\bar{\psi}^l_k D^k}{a-\lambda^l}(e^{a\tau}-e^{\lambda^l\tau}) + \frac{\bar{\psi}^l_k F^k}{2a-\lambda^l}(e^{2a\tau}-e^{\lambda^l\tau})  - \frac{\bar{\psi}^l_k E^k}{\lambda^l}(1-e^{\lambda^l\tau}) \right),
\ee 

with $D^k$, $E^k$ and $F^k$ being constant within the linearized field.


\subsection{Axisymmetric case}

In the previous sections the analytical solution for the particle coordinates as functions of time was derived. In the derivation the eigenvalues $\lambda^l$ and eigenvectors $\psi^i_l$ of the (3x3) matrix $a^i_l$ played an essential role in calculating the coordinates $x^i(\tau)$. From the form of the elements of $a^i_l$ one can deduce that for a general non-axisymmetric systemone one eigenvalue will always be equal to $0$. This is not problematic as long as the eigenvalue corresponds to a non-trivial eigenvector, which is the case. If one is interested in calculating the analytical solution for the coordinates in an axisymmetric (symmetric in $\varphi$) configuration, the additional symmetry will reduce the problem to a two-dimensional system and furthermore not allow to use the same derivation shown in the previous section. This occurs since the matrix $a^i_l$ then has two zero-valued eigenvalues which no longer have two linearly independent eigenvectors. In the derivation above the inverse of the matrix containing the eigenvectors $\psi^i_l$ was needed but since this matrix becomes singular when the eigenvectors are no longer linearly independent, a new approach is needed. In the upcoming sub-sections an analog solution for the axisymmetric case will be derived.

\subsection{Axisymmetric homogeneous solution} 
 This section presents the derivation of the analytical solution to the homogeneous part of equation \ref{standeqset} for the toroidally axisymmetric case. Here, all derivatives with respect to $\varphi$ are $0$ and it is furthermore assumed that no electric field is present. The matrix $a^i_l$ can then be written in the following notation omitting all zero-valued elements and introducing the abbreviations $d_{ij}=\frac{\partial h_i}{\partial x^j}$ and $u_i = \frac{\partial U}{\partial x^i}$:
\[
a^i_l= \frac{cm}{e}
\begin{pmatrix}
-d_{21}u_3 & 0 & -d_{23}u_3  \\
d_{31}u_1+d_{11}u_3 & 0 &-d_{33}u_1+d_{13}u_3  \\
d_{21}u_1& 0 &d_{23}u_1 
\end{pmatrix}
\]

Due to the linearization of the fields, the values for $d_{ij}$ and $u_i$ remain constant withing a given tetrahedron.
From the form of matrix $a^i_l$ can be deduced that the values for $\frac{\textrm{d}x_2}{\textrm{d}\tau}$ do not depend on $x_2$, the system of differential equations therefore reduces to a two-dimensional system, where the $x_2$-component can be calculated independently from the solutions for $x_1$ and $x_2$. The two-dimensional system of equations for the $x_1$ and $x_3$ component can be formulated using a reduced matrix 

\[
\tilde{a}^i_l= \frac{cm}{e}
\begin{pmatrix}
-d_{21}u_3 & - d_{23}u_3  \\
d_{21}u_1&d_{23}u_1 
\end{pmatrix}
\]

such that

\bea{eq:2d_system_explicit}
\begin{pmatrix}	\dot{x_1}(\tau)\\ \dot{x_3}(\tau)\end{pmatrix} = \frac{cm}{e} \begin{pmatrix}-d_{21}u_3 & -d_{23}u_3 \\d_{21}u_1&d_{23}u_1 \end{pmatrix} \cdot \begin{pmatrix}	x_1(\tau)\\ x_3(\tau)\end{pmatrix} + \begin{pmatrix} q_1(\tau)\\q_3(\tau) \end{pmatrix}.
\eea
In this reduced system of equations, the Ansatz 

\be{Ansatzeq21}
\vec{x} = e^{\lambda\tau}\vec{\psi}
\ee

will be used in order to construct the homogeneous solution, where $\lambda$ denotes the eigenvalues of $\tilde{a}^i_l$ and $\vec{\psi}$ the corresponding eigenvector. 

Using this, the homogeneous part of equation \ref{standeqset} yields the eigenvalue equation

\be{Ansatzeq22}
\lambda e^{\lambda \tau}\vec{\psi} = \hat{a} e^{\lambda\tau}\vec{\psi}
\ee



The general solution to the homogeneous differential equation can then be written as

\be{eq:homogeneousSolution1}
x_{(h)}^i(\tau) = C_1e^{\lambda_1\tau}\psi^i_1 + C_2e^{\lambda_2\tau}\psi^i_2.
\ee

Explicit calculation of the eigenvalues and corresponding eigenvectors of $a^i_l$ yields

\bea{eq:Eig(a^i_l)}
\lambda_{1} &=& 0\\
\lambda_{2} &=& \lambda = \frac{cm}{e}\left( d_{23}u_1 - d_{21}u_3 \right) \\
\psi^i_1 &=& \begin{pmatrix}
	\frac{-d_{23}}{d_{21}}\\
	1
\end{pmatrix}\\
\psi^i_2 &=& \begin{pmatrix}
	\frac{-u_{3}}{u_{1}}\\
	1
\end{pmatrix}
\eea

and the general solution to the homogeneous system therefore becomes

\be{eq:homogeneousSolution2}
x_{(h)}^i(\tau) = C_1\begin{pmatrix}\frac{-d_{23}}{d_{21}}\\1\end{pmatrix} + C_2e^{\frac{cm}{e}\left( d_{23}u_1 - d_{21}u_3 \right)\tau}\begin{pmatrix}\frac{-u_{3}}{u_{1}}\\1\end{pmatrix}
\ee


with the $C_i$ denoting arbitrary constants given by initial conditions of the problem. The eigenvectors furthermore need not be normalized since any normalization constant could be pulled into the $C_i$.

\subsection{Axisymmetric Particular Solution: Variation of Constants}

Now that the solution to the homogeneous part has been found, one is interested in a particular solution to the inhomogeneous differential equation \ref{standeqset} to construct the general solution. In order to do this, one can once more apply the method of Variation of Constants where coefficients $C_i$ of the homogeneous solution are treated as functions of $\tau$:
\be{eq:homogeneousSolution3}
x_{(p)}^i(\tau) = C_1(\tau)\psi^i_1 + C_2(\tau)e^{\lambda\tau}\psi^i_2
\ee

Calculating the derivative of this equation yields:
\be{Variation of Constants}
\dv{x_{(p)}^i}{\tau} = {C_1}^\prime(\tau) \psi^i_1 + {C_2}^\prime(\tau)e^{\lambda\tau}\psi^i_2 +C_2(\tau)\lambda e^{\lambda\tau}\psi^i_2
\ee

Inserting this expression into equation \ref{eq:2d_system_explicit} results to

\be{VoC2}
q^i(\tau)  = {C_1}^\prime(\tau)\psi^i_1 + {C_2}^\prime(\tau)e^{\lambda\tau}\psi^i_2.
\ee

It is convenient to write this expression as a matrix vector product, explicitly given as

\be{eq:matrix product for q}
\begin{pmatrix} q^1(\tau) \\ q^3(\tau) \end{pmatrix} = \underbrace{\begin{bmatrix} \psi^i_1,e^{\lambda \tau}\psi^i_2 \end{bmatrix}}_{M} \cdot 
\begin{pmatrix} {C_1}^\prime (\tau)  \\{C_2}^\prime (\tau) \end{pmatrix} = \begin{pmatrix}
	\frac{-d_{23}}{d_{21}} & \frac{-u_3}{u_1}e^{\lambda\tau} \\ 1 & e^{\lambda\tau}
\end{pmatrix}\cdot 
\begin{pmatrix} {C_1}^\prime (\tau)  \\{C_2}^\prime (\tau) \end{pmatrix}.
\ee

By inverting $M$ and multiplying by this inverse matrix from the left one obtains the explicit expression

\be{eq:explicit expression for Ci}
\begin{pmatrix} {C_1}^\prime (\tau)  \\{C_2}^\prime (\tau) \end{pmatrix}  = 
M^{-1} \cdot \begin{pmatrix} q^1(\tau) \\ q^3(\tau) \end{pmatrix} = \frac{cm}{e\lambda}\begin{pmatrix}
	-d_{21}u_1 & -d_{21}u_3 \\ d_{21}u1e^{-\lambda\tau} & d_{23}u_1e^{-\lambda \tau}
\end{pmatrix} \cdot \begin{pmatrix} q^1(\tau) \\ q^3(\tau) \end{pmatrix}
\ee

for ${C_i}^\prime(\tau)$.

One can now formally replace $\tau$ by $\tau^\prime$ integrate this expression from $0$ to $\tau$ since the initial conditions will be given for $\tau = 0$:
\be{C_i}
\begin{pmatrix}	C_1(\tau) \\ C_2(\tau)\end{pmatrix}  =  \frac{cm}{e\lambda} \int_0^\tau d\tau^\prime \begin{pmatrix}
	-d_{21}u_1 & -d_{21}u_3 \\ d_{21}u_1e^{-\lambda\tau^\prime} & d_{23}u_1e^{-\lambda \tau^\prime}
\end{pmatrix} \cdot \begin{pmatrix} q^1(\tau^\prime) \\ q^3(\tau^\prime) \end{pmatrix}\\
\ee

Next, one can continue by using the explicit expression for $q^i$, given in equation \ref{qi_veryshort}. Using the fact that $a = -\lambda$ simplifies this result to
\be{}
q^k(\tau) = e^{-\lambda\tau}D^k + e^{-2\lambda\tau}F^k + E^k
\ee

where each vectorial quantity consists only of the $x_1$ and $x_3$ components.

The results for $C_i(\tau)$ can then be evalutated to

\bea{Explicit C evaluation}
C_1(\tau) &=& \frac{cm}{e}d_{21}u_k\left[ \frac{e^{-2\lambda\tau}-1}{2\lambda^2} F^k + \frac{e^{-\lambda\tau}-1}{\lambda^2}D^k -\frac{\tau}{\lambda}E^k\right]\\
C_2(\tau) &=& \frac{cm}{e}u_1 d_{2k} \left[ \frac{1-e^{-3\lambda\tau}}{3\lambda^2} F^k + \frac{1-e^{-2\lambda\tau}}{2\lambda^2} D^k  + \frac{1-e^{-\lambda\tau}}{\lambda^2}E^k\right]\eea

with $u_k$ and $d_{2k}$ being
\bea{}
 u_k = \begin{pmatrix} u_1 \\ u_3 \end{pmatrix},& & d_{2k} = \begin{pmatrix} d_{21}\\ d_{23} \end{pmatrix}.
\eea

These expressions for $C_i(\tau)$ can now be put into equation \ref{eq:homogeneousSolution3} to calculate the particular solution. Superposition of particular and homogeneous solution constructs the general solution to the set of differential equations.

\bea{GeneralAxisymmetricSolution1}
x^i_{(g)}(\tau)= x^i_{(h)}+ x^i_{(p)} = (\tilde{C}_1 + C_1(\tau) ) \psi^i_1 + (\tilde{C}_2 + C_2(\tau)) e^{\lambda\tau}\psi^i_2
\eea

Since the $C_i(\tau)$ were integrated from $\tau^\prime = 0$ to $\tau\prime = \tau$ they vanish for $\tau = 0$, while initial conditions require that \be{}
x^i_{(g)}(\tau = 0) =  x^i_{(0)} = \tilde{C}_1 \psi^i_1 + \tilde{C}_2\psi^i_2 \hspace{0.1 cm}.
\ee


Explicitly, this gives two equations for $\tilde{C}_i$, namely
\bea{}
x^1_{(0)} &=& -\frac{d_{23}}{d_{21}}\tilde{C}_1-\frac{u_3}{u_1} \tilde{C}_2 \text{,}\\
x^3_{(0)} &=& \tilde{C}_1+\tilde{C}_2 \text{.}
\eea
Using $\frac{cm}{e}\lambda = (d_{23}u_1-d_{21}u_3)$, the $\tilde{C}_i$ are therefore
\bea{}
\tilde{C}_1 &=& -\frac{cm}{e}d_{21} \frac{u_ix^i_{(0)}}{\lambda} \text{ , }\\
\tilde{C}_2 &=& \frac{cm}{e}u_1 \frac{d_{2i}x^i_{(0)}}{\lambda} \hspace{0.1 cm}\text{. }
\eea

Inserting this into equation \ref{GeneralAxisymmetricSolution1} yields the analytical result for the coordinates $x^1, x^3$:
\bea{AxisymmetricAnalyticalExpressionX}
x^1(\tau) &=& \frac{cm}{e\lambda}\bigg[ x^k_{(0)}\left(d_{23}u_k-u_3d_{2k}e^{\lambda\tau} \right) \nonumber \\
&-& d_{23}u_k \left( \frac{e^{-2\lambda\tau}-1}{2\lambda}F^k + \frac{e^{-\lambda\tau}-1}{\lambda}D^k - \tau E^k \right) \nonumber \\
&-&  u_3d_{2k}\left( \frac{e^{\lambda\tau}-e^{-2\lambda\tau}}{3\lambda}F^k + \frac{e^{\lambda\tau}-e^{-\lambda\tau}}{2\lambda}D^k+\frac{e^{\lambda\tau}-1}{\lambda}E^k\right)\bigg] \\
x^3(\tau) &=& \frac{cm}{e\lambda}\bigg[ x^k_{(0)}\left(-d_{21}u_k+u_1d_{2k}e^{\lambda\tau} \right) \nonumber\\
&+& d_{21}u_k \left( \frac{e^{-2\lambda\tau}-1}{2\lambda}F^k + \frac{e^{-\lambda\tau}-1}{\lambda}D^k - \tau E^k \right) \nonumber\\
&+& u_1d_{2k}\left( \frac{e^{\lambda\tau}-e^{-2\lambda\tau}}{3\lambda}F^k + \frac{e^{\lambda\tau}-e^{-\lambda\tau}}{2\lambda}D^k+\frac{e^{\lambda\tau}-1}{\lambda}E^k\right)\bigg] \eea

Now that $x^1$ and $x^3$ have been found, $x^2$ can be calculated via the second component of the differential equation set using these newly found solutions.
The second component of the differential equation can be written as

\be{}
\dot{x}^2(\tau) =  a_{21}x^1(\tau) + a_{23}x^3(\tau) + q^2(\tau) \hspace{0.1 cm},
\ee

which is explicitly given as

\be{}
\dot{x}^2(\tau) = \frac{cm}{e} \left(d_{31}u_1+d_{11}u_3 \right)x^1(\tau) + \frac{cm}{e}\left(-d_{33}u1 + d_{13}u_3 \right)x^3(\tau) + e^{-\lambda\tau}D^2 + e^{-2\lambda\tau}F^2 + E^2 \hspace{0.1 cm}.
\ee

Formally replacing $\tau$ by $\tau^\prime$ and subsequent integration over $\tau^\prime$ from $0$ to $\tau$ yields the result for $x^2(\tau)$. Since this equation depends only on $\dot{x}^2(\tau)$ and not on $x^2(\tau)$, an arbitrary constant $C$ can be added to $x^2(\tau)$ which will be determined by initial conditions. Since the integral over $\tau^\prime$ starts at $0$, this constant will be given by $C = x^2_{(0)}$.

For clarity, one can define that
\bea{x2-AxisymmetricAnalyticalSolution}
\textrm{X}^1(\tau) &=& \int_{0}^{\tau} x^1(\tau^\prime)d\tau^\prime \\
&=& \frac{cm}{e\lambda}\bigg[ x^k_{(0)}\left(d_{23}u_k\tau-u_3d_{2k}\frac{e^{\lambda\tau}-1}{\lambda} \right) \nonumber\\
&-& d_{23}u_k \left( \frac{e^{-2\lambda\tau}+2\lambda\tau-1}{4\lambda^2}F^k - \frac{e^{-\lambda\tau}+\lambda\tau-1}{\lambda^2}D^k - \frac{\tau^2}{2} E^k \right) \nonumber\\
&-&  u_3d_{2k}\left( \frac{2e^{\lambda\tau}+e^{-2\lambda\tau}-3}{6\lambda^2}F^k + \frac{e^{\lambda\tau}+e^{-\lambda\tau}-2}{2\lambda^2}D^k+\frac{e^{\lambda\tau}-\lambda\tau-1}{\lambda^2}E^k\right)\bigg],\nonumber\\
\textrm{X}^3(\tau) &=& \int_{0}^{\tau} x^3(\tau^\prime)d\tau^\prime \\
&=& \frac{cm}{e\lambda}\bigg[ x^k_{(0)}\left(-d_{21}u_k\tau+u_1d_{2k}\frac{e^{\lambda\tau}-1}{\lambda} \right) \nonumber\\
&+& d_{21}u_k \left( \frac{e^{-2\lambda\tau}+2\lambda\tau-1}{4\lambda^2}F^k - \frac{e^{-\lambda\tau}+\lambda\tau-1}{\lambda^2}D^k - \frac{\tau^2}{2} E^k \right) \nonumber\\
&+&  u_1d_{2k}\left( \frac{2e^{\lambda\tau}+e^{-2\lambda\tau}-3}{6\lambda^2}F^k + \frac{e^{\lambda\tau}+e^{-\lambda\tau}-2}{2\lambda^2}D^k+\frac{e^{\lambda\tau}-\lambda\tau-1}{\lambda^2}E^k\right)\bigg],\nonumber\\
\textrm{Q}^2(\tau) &=& \int_{0}^{\tau} q^2(\tau^\prime)d\tau^\prime \\
&=& \frac{1-e^{-\lambda\tau}}{\lambda}D^2 + \frac{1-e^{-\lambda\tau}}{2\lambda}F^2 + \tau E^2 \nonumber\hspace{0.1 cm}.
\eea

The solution for $x^2(\tau)$ can then be compactly written as

\be{}
x^2(\tau) = \frac{cm}{e} \left(d_{31}u_1+d_{11}u_3 \right)\textrm{X}^1(\tau) + \frac{cm}{e}\left(-d_{33}u_1 + d_{13}u_3 \right)\textrm{X}^3(\tau) + \textrm{Q}^2(\tau)\hspace{0.1 cm}.
\ee





\clearpage
\section{Integration of equations of motion with Runge-Kutta method of $4^{\text{th}}$-order}

\subsection{Derivation of the RK4-Error for the geometric integrator}
\noindent
Our set of 4 linear differential equations for 4 unknowns $z^i$ is
\be{}
\frac{\rd z^i}{\rd \tau} = a^i_l z^l + b^i,
\ee
where $a^i_l$ is constant 4$\times$4 matrix and $b^i$ is constant 4D vector.
It is convenient to use here vector notation for unknowns $\bz$, for matrix $\hat \ba$ and for r.h.s. vector $\bb$
and to denote tensor convolutions with $\cdot$
\be{matrnot}
\frac{\rd \bz}{\rd \tau} = \hat \ba \cdot \bz +\bb.
\ee
We denote starting value of $\bz$ for $\tau=0$ with $\bz_0$, $\bz(0)=\bz_0$  and look for the solution at time moment $\tau$
in the form of time series
\be{timser}
\bz=\bz_0 + \tau \bz_1 + \tau^2 \bz_2 + ... = \sum\limits_{k=0}^\infty \tau^k \bz_k,
\ee
where $\bz_k$ are constant vectors.
Substituting this in~\eq{matrnot} we get
\be{get}
\sum\limits_{k=0}^\infty (k+1)\tau^k \bz_{k+1} = \bb + \sum\limits_{k=0}^\infty \tau^k \hat\ba\cdot\bz_k.
\ee
Equating independently different powers of $\tau$ we get an infinite chain
\bea{chain}
\bz_1 &=& \bb + \hat\ba\cdot\bz_0,
\nonumber \\
2\bz_2 &=& \hat\ba\cdot\bz_1,
\nonumber \\
3\bz_3 &=& \hat\ba\cdot\bz_2,
\nonumber \\
\dots,
\nonumber \\
k\bz_k &=& \hat\ba\cdot\bz_{k-1},
\nonumber \\
\dots
\eea
which can be solved as follows
$$
\bz_k = \frac{1}{k!}\hat \ba^{k-1}\cdot\bz_1,
$$
where $\hat \ba^n$ denotes product of n-matrices $\hat \ba$. Thus, explicitly series~\eq{timser} is
\be{explser}
\bz=\bz_0 + \sum\limits_{k=1}^\infty \frac{\tau^k}{k!} \left(\hat \ba^{k-1}\cdot\bb + \hat \ba^k\cdot \bz_0\right).
\ee
This series converges for any matrix $\hat \ba$ and for any value of $\tau$ because of the factorial in denominator.

\noindent
Let us introduce a more general notation for autonomous sets of equations
\be{autonom}
\frac{\rd \bz}{\rd \tau} = \bbf(\bz),
\ee
where $\bbf(\bz)$ is generally a nonlinear vector function of $\bz$, and ``autonomous'' means that $\bbf$ does not depend on time 
explicitly. In our case of a linear set~\eq{matrnot} this function is
\be{fdef}
\bbf(\bz) = \bb + \hat \ba \cdot\bz.
\ee
Denoting now with $\bz_{RK4}$ a result of a single 4-th order Runge Kutta step from 0 to $\tau$ with initial value $\bz_0$, 
explicitly this result is
\bea{RK4}
\bz_{RK4} &=& \bz_0 + \frac{\tau}{6}\bbf(\bz_0)+\frac{\tau}{3}\bbf\left(\bz_0 + \frac{\tau}{2}\bbf(\bz_0)\right)
+\frac{\tau}{3}\bbf\left(\bz_0 + \frac{\tau}{2}\bbf\left(\bz_0 + \frac{\tau}{2}\bbf(\bz_0)\right)\right)
\nonumber \\
&+&
\frac{\tau}{6}\bbf\left(\bz_0 + h \bbf\left(\bz_0 + \frac{\tau}{2}\bbf\left(\bz_0 + \frac{\tau}{2}\bbf(\bz_0)\right)\right)\right).
\eea
Denoting now $\bbf_0=\bbf(\bz_0)$ we get in accordance with~\eq{fdef}
\be{f0not}
\bbf(\bz_0 + \delta \bz) = \bbf_0 + \hat\ba\cdot \delta \bz.
\ee
Using this formula in~\eq{RK4} we can make all substitutions and get
\be{RK4lin}
\bz_{RK4} = \bz_0 + \tau \bbf_0 + \frac{\tau^2}{2}\hat\ba\cdot\bbf_0 + \frac{\tau^3}{6}\hat\ba\cdot\hat\ba\cdot\bbf_0 
+\frac{\tau^4}{24}\hat\ba\cdot\hat\ba\cdot\hat\ba\cdot\bbf_0
= \bz_0 + \sum_{k=1}^4 \frac{\tau^k}{k!}\hat\ba^{k-1}\cdot\bbf_0
\ee
Using the same notation for $\bbf_0 = \bb + \hat \ba \cdot\bz_0 $ in the exact solution~\eq{explser} we can present this solution
in the form
\be{explser_renot}
\bz=\bz_0 + \sum\limits_{k=1}^\infty \frac{\tau^k}{k!}\ba^{k-1}\cdot \bbf_0.
\ee
Thus, 4-th order Runge Kutta solution~\eq{RK4lin} differs from the exact solution~\eq{explser_renot} by the terms
of the order $\tau^5 \ba^4$ and higher. Therefore, our solution to the truncated set ignoring the FLR effects, which has parabolic
dependence on time, is described by the Runge Kutta solution solution exactly (its exact series~\eq{explser_renot} has only three first
terms which are also fully included in~\eq{RK4lin}).

\subsection{Taylor series expansion of analytical solution}
The analytical solution can be written as
\be{}
x^i(\tau) =   \psi^i_l\bar{\psi}^l_k  \left( x^k_{(0)}e^{\lambda^l \tau} + \frac{ D^k}{a-\lambda^l}(e^{a\tau}-e^{\lambda^l\tau}) + \frac{ F^k}{2a-\lambda^l}(e^{2a\tau}-e^{\lambda^l\tau})  + \frac{E^k}{\lambda^l}(e^{\lambda^l\tau}-1) \right),
\ee
whereas the Runge-Kutta method of $4^{\text{th}}$-order computes the solution exactly up to the $4^{\text{th}}$-order of a Taylor series expansion:
\be{taylor_x-RK}
\begin{split}
x^i_{\text{RK4}} =  &\psi^i_l\bar{\psi}^l_k \bigg[\bigg. x^k_{(0)} + \tau\left( \lambda^l x^k_{(0)} + D^k + F^k + E^k \right)  \\
& +\frac{\tau^2}{2}\left(   (\lambda^l)^2 x^k_{(0)} + (a + \lambda^l)D^k + (2a+\lambda^l)F^k + \lambda^lE^k  \right)\\
& +\frac{\tau^3}{6} \left((\lambda^l)^3 x^k_{(0)} + (a^2 + \lambda^l(a+\lambda^l))D^k + (4a^2+\lambda^l(2a+\lambda^l))F^k +(\lambda^l)^2E^k \right)\\
& +\frac{\tau^4}{24}\left( (\lambda^l)^4 x^k_{(0)} + (a+\lambda^l)(a^2 + (\lambda^l)^2)D^k + (2a+\lambda^l)(4a^2+ (\lambda^l)^2)F^k + (\lambda^l)^3E^k \right) \bigg]
%	x^i_{\text{RK4}} &=  \sum_{l = 1}^{3} \psi^i_l\bar{\psi}^l_i \left(\sum_{j = 0}^{4}  \left[ \frac{(\lambda^l\tau)^j }{j!} %\left(   x^i_{(0)} -  \frac{ D^k}{a-\lambda^l}  - \frac{ F^k}{2a-\lambda^l}  \right)  \right. \right. \\
%	&\left. \left. + \frac{(a\tau)^j}{j!} \frac{ D^k}{a-\lambda^l}  + \frac{(2a\tau)^j}{j!}  \frac{ F^k}{2a-\lambda^l}  \right] - %\sum_{j = 1}^{4}  \frac{(-\lambda^l)^{j-1}\tau^j }{j!}  E^k \right) \\
\end{split}
\ee

By substituting $D^k$,$E^k$ and $F^k$ with their original values in terms of $\alpha^k$, $\beta^k$ and $\gamma^k$, respectively, one can avoid numerical cancelation, which unfortunately occurs. Formula \ref{taylor_x-RK} can then be rewritten as

\be{taylor_x-RK2}
\begin{split}
	x^i_{\text{RK4}} =  &\psi^i_l\bar{\psi}^l_k \bigg[\bigg. x^k_{(0)} + \tau\left( \lambda^l x^k_{(0)} + q^k_{(0)} \right)  \\
	& +\frac{\tau^2}{2}\left((\lambda^l)^2 x^k_{(0)} + \lambda^l q^k_{(0)} +  (v_{\parallel,0}+\frac{b}{a})(a\alpha^k + 2v_{\parallel,0}\beta^k) \right)\\
		& +\frac{\tau^3}{6} \bigg(\bigg.(\lambda^l)^3 x^k_{(0)} + (\lambda^l)^2 q^k_{(0)}+ (av_{\parallel,0}+b)(a+\lambda^l)\alpha^k  \\ 
		&+(-2b(a+\lambda^l)(v_{\parallel,0}+\frac{b}{a}) + 2a(2a+\lambda^l)(v_{\parallel,0}+\frac{b}{a})^2)\beta^k \bigg.\bigg)\\
	& +\frac{\tau^4}{24}\bigg(\bigg.(\lambda^l)^4 x^k_{(0)} + (\lambda^l)^3 q^k_{(0)} + a(a^2+\lambda^la+\lambda^2)(v_{\parallel,0}+\frac{b}{a})\alpha^k \\
	& +(- 2 \frac{b}{a}(v_{\parallel,0}+\frac{b}{a}) + 
	a(8a^2+4\lambda^la + 2(\lambda^l)^2)(v_{\parallel,0}+\frac{b}{a})^2)\beta^k
	 \bigg.\bigg) \bigg]
	%	x^i_{\text{RK4}} &=  \sum_{l = 1}^{3} \psi^i_l\bar{\psi}^l_i \left(\sum_{j = 0}^{4}  \left[ \frac{(\lambda^l\tau)^j }{j!} %\left(   x^i_{(0)} -  \frac{ D^k}{a-\lambda^l}  - \frac{ F^k}{2a-\lambda^l}  \right)  \right. \right. \\
	%	&\left. \left. + \frac{(a\tau)^j}{j!} \frac{ D^k}{a-\lambda^l}  + \frac{(2a\tau)^j}{j!}  \frac{ F^k}{2a-\lambda^l}  \right] - %\sum_{j = 1}^{4}  \frac{(-\lambda^l)^{j-1}\tau^j }{j!}  E^k \right) \\
\end{split}
\ee


If one is interested in the absolute error of this method, further terms (up to infinity) of the Taylor series expansion of the solution can be computed:
\be{}
\begin{split}
	\Delta x^i_{(5)} &= \psi^i_l \sum_{l = 1}^{3} \left( \sum_{j = 5}^{\infty}  \left[ \frac{(\lambda^l\tau)^j }{j!} \left(  \bar{\psi}^l_i x^i_{(0)} -  \frac{\bar{\psi}^l_k D^k}{a-\lambda^l}  - \frac{\bar{\psi}^l_k F^k}{2a-\lambda^l}  \right)  \right. \right. \\
	&\left. \left. + \frac{(a\tau)^j}{j!} \frac{\bar{\psi}^l_k D^k}{a-\lambda^l}  + \frac{(2a\tau)^j}{j!}  \frac{\bar{\psi}^l_k F^k}{2a-\lambda^l}  \right] - \sum_{j = 5}^{\infty}  \frac{(-\lambda^l)^{j-1}\tau^j }{j!} \bar{\psi}^l_k E^k \right) \\
\end{split}
\ee
An expression for the Runge-Kutta error of position as a function of initial conditions, eigenvalues and eigenvectors of $\hat{a}$ has therefore been found.




%\subsection{Physical estimation by analytical simplification}
%\subsubsection{Eigenvalues of $\hat{a}$}
%\noindent
%In this section we want to calculate an analytical expression for the eigenvalues of $\hat{a}$ which we can use to calculate the RK-error according to formula \ref{taylor_x-RK}. It is important to note, that we only need to calculate the eigenvalues for the matrix $a_l^i$ with $l = 1,2,3$, as the $v_\parallel$ -dependency is contained in function $q^i$. Let's start by looking at the definition of $a^i_l$ (equation \ref{amatdef}): 
%$ a^i_l = \varepsilon^{ijk}\left(
%2\difp{U}{x^l}\difp{}{x^j}\frac{B_k}{\omega_c}+\difp{U}{x^j}\difp{}{x^l}\frac{B_k}{\omega_c}
%\right) $
%We can see that this function depends on the gradient of $U$, which in the general case will point in some unknown direction. Since eigenvalues are invariant under unitary transformations, we can simply rotate the coordinate system in such a way that we can assume that the gradient of $U$ is pointing in the $x^1$ direction, this will simplify the calculation for the eigenvalues of $a^i_l$ tremendously. The gradient of $U$ is therefore:
%\be{nablaU}
%\vec{\nabla}U = \vec{e_1} \frac{\partial U}{\partial x^1}
%\ee
%Let's try to write $a^i_l$ in a compact way by using $\frac{B_k}{\omega_c} = h_k \frac{cm}{e}$ and setting $\frac{cm}{e} = 1$:
%\be{a_easy}
%\hat{a}^\prime = \vec{\nabla} (\underbrace{\vec{h}\cross\vec{\nabla} U}_{\vec{v}}) 
%\ee
%$\hat{a}^\prime$ then takes the form:
%\be{amat1}
%\hat{a}^\prime=
%\begin{bmatrix}
%0 & \frac{\partial v_2}{\partial x^1}  & \frac{\partial v_3}{\partial x^1} \\
%0 & \frac{\partial v_2}{\partial x^2}  & \frac{\partial v_3}{\partial x^2} \\
%0 & \frac{\partial v_2}{\partial x^3}  & \frac{\partial v_3}{\partial x^3}
%\end{bmatrix}
% = 
% \begin{bmatrix}
% 	0 & \alpha_{12}  &\alpha_{13}  \\
% 	0 & \alpha_{22}   &\alpha_{23}   \\
% 	0 & \alpha_{32}   & \alpha_{33} 
% \end{bmatrix}
%\ee
%We are now interested in the calculation of the eigenvalues of this matrix:
%\be{a_eigenvalues1}
%det\begin{vmatrix}
%	0 -\lambda & \frac{\partial v_2}{\partial x^1}  & \frac{\partial v_3}{\partial x^1} \\
%	0 & \frac{\partial v_2}{\partial x^2} -\lambda  & \frac{\partial v_3}{\partial x^2} \\
%	0 & \frac{\partial v_2}{\partial x^3}  & \frac{\partial v_3}{\partial x^3}-\lambda 
%\end{vmatrix} \overset{!}{=} 0
%\ee
%The equation for the eigenvalues is therefore:
%\be{a_eig2}
%(-\lambda)(\alpha_{22}-\lambda)(\alpha_{33}-\lambda) + \alpha_{32}\alpha_{23}\lambda = 0 
%\ee
%Expanding the products yields:
%\be{a_eig3}
%-\lambda^3 + \lambda^2(\alpha_{33}+\alpha_{22}) - \lambda(\alpha_{22}\alpha_{33}-\alpha_{23}\alpha_{32}) = 0
%\ee
%We can immediately say, that one eigenvalue is $\lambda = 0$, for the other eigenvalues we get:
%\be{a_eig4}
%\lambda_{2,3} = \frac{\alpha_{22}+\alpha_{33} \pm \sqrt{(\alpha_{22}+ \alpha_{33})^2 - 4(\alpha_{22}\alpha_{33}-\alpha_{23}\alpha_{32})}}{2}
%\ee
%After further simplification this formula can be written as:
%\be{a_eig5}
%\lambda_{2,3} = \frac{\alpha_{22}+\alpha_{33} \pm \sqrt{(\alpha_{22}- \alpha_{33})^2 + 4\alpha_{23}\alpha_{32}}}{2}
%\ee

%\subsubsection{Physical estimation of Eigenvalues}
%\noindent
%Starting with matrix \ref{eq:matrix3}, one can assume, that $\frac{\partial U}{\partial x^j}$ points only in $\vec{e}_1$ direction.
%\be{}
%\frac{\partial U}{\partial x^j} = \vec{e}_1 \frac{\partial U}{\partial x^1}.
%\ee
%Thus, the matrix elements $a^i_1$ are zero.\\
%By using $\frac{B_k}{\omega_c} = h_k \frac{cm}{e}$, the Eigenvalues of $a^i_l$ are:
%\bea{}
%\lambda_1 &=& 0 \nonumber \\
%\lambda_2 &=& \frac{1}{2} \frac{cm}{e} \left( \frac{\partial h_2}{\partial x^3} - \frac{\partial h_3}{\partial x^2} +\sqrt{\left( \frac{\partial h_2}{\partial x^3}  + \frac{\partial h_3}{\partial x^2}   \right)^2  - 4 \frac{\partial h_2}{\partial x^2} \frac{\partial h_3}{\partial x^3} } \right) \frac{\partial U}{\partial x^1} \nonumber \\
%\lambda_3 &=& \frac{1}{2} \frac{cm}{e} \left( \frac{\partial h_2}{\partial x^3} - \frac{\partial h_3}{\partial x^2} -\sqrt{\left( \frac{\partial h_2}{\partial x^3}  + \frac{\partial h_3}{\partial x^2}   \right)^2  - 4 \frac{\partial h_2}{\partial x^2} \frac{\partial h_3}{\partial x^3} } \right) \frac{\partial U}{\partial x^1}
%\eea
%Since we assume an axisymmetric magnetic configuration (unperturbed Tokamak), the magnetic field is invariant regarding rotation around the z-axis. Hence, all derivatives $\frac{\partial}{\partial x^2}$ are zero.\\
%One obtains only one non-zero eigenvalue
%\be{}
%\lambda_2 = \frac{cm}{e} \frac{\partial h_2}{\partial x^3} \frac{\partial U}{\partial x^1}.
%\ee
%\textbf{Physical estimation of $\frac{\partial h_2}{\partial x^3}$:}\\
%First of all, we want a physical estimation for $\frac{\partial h_2}{\partial x^3}$.
%\be{}
%\frac{\partial h_2}{\partial x^3} = \frac{\partial h_\varphi}{\partial z} =  \frac{\partial }{\partial z} \frac{B_\varphi}{B} = \frac{\partial }{\partial z} \frac{\hat{B} \cdot \vec{e}_\varphi }{B}
%\ee
%For the covariant component of the magnetic field $\hat{B}$, one needs the covariant unit vector 
%\be{}
%\vec{e}_\varphi = \frac{\mathrm{d}\vec{r}}{\mathrm{d} \varphi} = R \hat{\vec{e}}_\varphi,
%\ee
%where $\hat{\vec{e}}_\varphi$ is the physical $\varphi$-unit vector in cylindrical coordinates.\\
%Thus, one obtains
%\bea{}
%\frac{\partial h_\varphi}{\partial z} &=& R \frac{\partial }{\partial z} \left( \frac{1}{B} \underbrace{ \hat{B} \cdot \hat{\vec{e}}_\varphi  }_{\hat{B}_{\text{tor}}} \right) = R \frac{\partial }{\partial z} \left( \frac{\hat{B}_{\text{tor}}}{\sqrt{\hat{B}_{\text{tor}}^2 + \hat{B}_{\text{pol}}^2}} \right) \nonumber \\
%&=& R \frac{\partial }{\partial z} \left( 1 - \frac{1}{2}  \frac{\hat{B}^2_{\text{pol}}}{\hat{B}^2_{\text{tor}}}\right) = -R \underbrace{\frac{\hat{B}_{\text{pol}}}{\hat{B}_{\text{tor}}}}_{\frac{r}{qR}} \frac{\partial }{\partial z} \frac{\hat{B}_{\text{pol}}}{\hat{B}_{\text{tor}}} = -\frac{r}{q} \frac{\partial }{\partial z} \frac{r}{qR}.
%\eea
%Expressing the minor radius r as $r = \sqrt{(R-R_0)^2+z^2}$ and computing the derivative $\frac{\partial r}{\partial z} = \frac{z}{r}$ yields
%\be{}
%\frac{\partial h_\varphi}{\partial z} = - \frac{\slashed{r}}{q^2R} \frac{z}{\slashed{r}} \sim \frac{r}{Rq^2}
%\ee
%\textbf{Physical estimation of $\frac{\partial U}{\partial x^1}$:}\\
%The quantity $U$ is defined by 
%\be{}
%U = \frac{v_\parallel^2}{2} = \frac{1}{m}\left(w-J_\perp \omega_c - e_\alpha \Phi  \right),
%\ee
%where $w$ and $J_\perp$ are constants. Therefore, (according to Sergei) the gradient of U can be estimated by
%\be{}
%\nabla U \sim \nabla B\frac{v_\perp^2}{B} \sim \frac{v^2}{L_B},
%\ee
%where $L_B$ is the characteristic length of the magnetic field and further $L_B \approx R$, where $R$ is the major radius.
%Thus, $\frac{\partial U}{\partial x^1} \approx \frac{v^2}{R} $.\\
%\textbf{Physical estimation of $\lambda_2$:}\\
%Putting together the two expressions from above yields
%\be{}
%\lambda_2 = \frac{cm}{e} \frac{\partial h_\varphi}{\partial z} \frac{\partial U}{\partial R}\sim  \frac{cm}{e} \frac{r}{Rq^2}  \frac{v^2}{R} = \frac{cm}{e} \frac{rv^2}{R^2q^2}.
%\ee
%
%\subsubsection{Physical estimation of $\Delta\tau$}
%\noindent
%The orbit parameter $\tau$ is related to time by
%\be{}
%\mathrm{d}t = B_\parallel^\ast \sqrt{g}\mathrm{d}\tau.
%\ee
%For a finite time $\Delta t$ to pass a tetrahedron toroidally and an estimation of $B_\parallel^\ast \approx B$ one obtains
%\be{}
%\Delta \tau \sim \frac{\Delta t}{BR} \sim \frac{1}{BR} \frac{\Delta l_\varphi}{v_\parallel} = \frac{1}{BR} \frac{\Delta l_\varphi}{\lambda_p v}, 
%\ee
%where $\Delta l_\varphi$ is the toroidal length of a tetrahedron, and $\lambda_p = \frac{v_\parallel}{v}$ is the pitch parameter.\\
%The finite time $\Delta t$ can be approximated by assuming that the toroidal length of a tetrahedron $\Delta l_\varphi$ is
%\be{}
%\Delta l_\varphi = \frac{2\pi}{N_\varphi}R,
%\ee
%where $N_\varphi$ is the number of cells in toroidal direction (1-dimensional).
%Thus, finite orbit parameter to pass a tetrahedron toroidally is
%\be{}
%\Delta \tau \sim \frac{2 \pi R}{\lambda_p v N_\varphi B R } = \frac{2 \pi}{\lambda_p v N_\varphi B} 
%\ee
%
%\subsubsection{Physical estimation of $\lambda_2 \Delta \tau$}
%\noindent
%The RK4 error of 3DGeoInt scales with the product $\lambda_2 \Delta \tau$:
%\be{}
%\lambda_2 \Delta\tau \sim \frac{cm}{e} \frac{rv^2}{R^2q^2} \frac{2 \pi}{\lambda_p v N_\varphi B}
%\ee
%By introducing the cyclotron frequency $\omega_c = \frac{eB}{mc}$, the above equation becomes
%\be{}
%\lambda_2 \Delta\tau \sim \frac{r}{R^2q^2} \frac{2 \pi}{\lambda_p N_\varphi} \underbrace{\frac{v}{\omega_c}}_{\approx \rho_l},
%\ee
%where $\rho_l$ is the larmor radius.
%
%\subsection{Relative error of the RK4-method}
%\noindent
%We are interested now, how large the relative error of the RK4-method is respective $N_\varphi$, $\rho_l$ and $\lambda_p$.
%Thus, we need to compute the following quantities:
%\bea{}
%\frac{\Delta R^{(5)}}{R^{\text{RK4}}} & \sim  &  \\
%\frac{\Delta \varphi^{(5)}}{\varphi^{\text{RK4}}} & \sim  & \\
%\frac{\Delta Z^{(5)}}{R^{\text{RK4}}} & \sim  & \\
%\frac{\Delta v_\parallel^{(5)}}{v_{\text{mod}}^{\text{RK4}}} & \sim  & 
%\eea
%%\end{document}

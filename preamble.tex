
\documentclass[a4paper, 12pt, numbers=noenddot, chapterprefix=true, appendixprefix=true]{report}


\usepackage[english]{babel}
\usepackage[utf8]{inputenc}
\usepackage[T1]{fontenc}

\usepackage{fancyhdr}
\usepackage[a4paper, bindingoffset=4mm, left=27mm, right=27mm, top=30mm, bottom=30mm]{geometry}


\setlength{\parindent}{0cm}
\usepackage[onehalfspacing]{setspace}
\usepackage{microtype}


\usepackage{amsmath,amssymb,amstext}[fleqn]
\usepackage{float}
%\usepackage{subfig}
\usepackage{subcaption}
\usepackage{graphicx}
\usepackage{picinpar}
\usepackage{wrapfig}
\usepackage{hyperref} % Für die Verwendung von URLs
\usepackage{gensymb}
%\usepackage[automark]{scrpage2}
\usepackage[table,xcdraw]{xcolor}
%\usepackage[stable]{footmisc} % Für Fußnoten in Sections
\usepackage[normalem]{ulem}


\usepackage{multirow}
\usepackage{tikz}
\usepackage{pdfpages}





\usepackage{listings}		% Notwendig für ff++listings
%\usepackage{ff++listings}	% ff++listings ermöglicht die Darstellung von FreeFEM++-Code


%\usepackage{geometry} % Seitenränder bearbeiten

\usepackage{pgfplots} % Für die Verwendung von Plots mit TIKZ
\pgfplotsset{compat=1.10} 

\usetikzlibrary{external} 	% Die PGF-Plots mit TIKZ werden extern berechnet
\tikzexternalize[prefix=tikz/]	% Die berechneten Plots werden in den Ordner "tikz" gespeichert
% Für die externe Berechnung wird lualatex verwendet.
\tikzset{external/system call={lualatex \tikzexternalcheckshellescape -halt-on-error -interaction=batchmode -jobname "\image" "\texsource"}}



%\geometry{a4paper,left=2.7cm,right=2.7cm, top=4cm, bottom=4cm}
\pagestyle{fancy}
\fancypagestyle{MyStyle}{%
	\fancyhead{} %Clean headers
	\fancyhead[RO]{\slshape\nouppercase{\leftmark}}
	\fancyhead[LE]{\slshape\nouppercase{\leftmark}}
	\renewcommand{\chaptermark}[1]{\markboth{ {\slshape{##1}}}{}}
}

\fancypagestyle{plain}{\fancyhead{} %Clean headers
	\fancyhead[RO]{\slshape\nouppercase{\leftmark}}
	\fancyhead[LE]{\slshape\nouppercase{\leftmark}}
	\renewcommand{\chaptermark}[1]{\markboth{ {\slshape{##1}}}{}}}

\pagestyle{fancy} %eigener Seitenstil
\lhead{}
\chead{}
%\fancyhf{} %alle Kopf- und Fußzeilenfelder bereinigen
%\fancyhead[L]{Titel} %Kopfzeile links
%\fancyhead[C]{} %zentrierte Kopfzeile
%\fancyhead[R]{Name} %Kopfzeile rechts
\renewcommand{\headrulewidth}{0.4pt} %obere Trennlinie
%\fancyfoot[C]{\thepage} %Seitennummer
\renewcommand{\footrulewidth}{0.4pt} %untere Trennlinie


\usepackage{etoolbox}
\makeatletter
\patchcmd{\@makechapterhead}{50\p@}{0pt}{}{}
\patchcmd{\@makeschapterhead}{50\p@}{0pt}{}{}
\makeatother





%%LUKAS PREAMBLE
\graphicspath{{figures/}{../figures/}}
\usepackage{siunitx}
\usepackage{caption}
\captionsetup[table]{singlelinecheck=false}
\usepackage{subcaption}
\usepackage{listings}		% Notwendig für ff++listings
\usepackage{pgfplots} % Für die Verwendung von Plots mit TIKZ
\pgfplotsset{compat=1.10} 
\usepackage{amsmath,xparse}
\usepackage{subfiles}
\usepackage{bm}
\usepackage{mathrsfs}
\usepackage{csquotes}
